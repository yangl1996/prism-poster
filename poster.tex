% Gemini theme
% https://github.com/anishathalye/gemini

\documentclass[final]{beamer}

% ====================
% Packages
% ====================

\usepackage[T1]{fontenc}
\usepackage{lmodern}
\usepackage[size=custom,width=120,height=72,scale=1.0]{beamerposter}
\usetheme{gemini}
\usecolortheme{gemini}
\usepackage{subcaption}
\usepackage{tikz}
\usepackage{gnuplot-lua-tikz}
\usepackage{graphicx}
\usepackage{booktabs}
\usepackage{tikz}
\usepackage{pgfplots}

% ====================
% Lengths
% ====================

% If you have N columns, choose \sepwidth and \colwidth such that
% (N+1)*\sepwidth + N*\colwidth = \paperwidth
\newlength{\sepwidth}
\newlength{\colwidth}
\setlength{\sepwidth}{0.025\paperwidth}
\setlength{\colwidth}{0.3\paperwidth}

\newcommand{\separatorcolumn}{\begin{column}{\sepwidth}\end{column}}

% ====================
% Title
% ====================

\title{Prism: Scaling Bitcoin by 10,000$\boldsymbol{\times}$}

\author{Lei Yang \inst{1} \and Vivek Bagaria \inst{2} \and Gerui Wang \inst{3} \and Mohammad Alizadeh \inst{1} \and David Tse \inst{2} \and Giulia Fanti \inst{4} \and Pramod Viswanath \inst{3}}

\institute[shortinst]{\inst{1} MIT CSAIL \samelineand \inst{2} Stanford University \samelineand \inst{3} University of Illinois at Urbana-Champaign \samelineand \inst{4} Carnegie Mellon University}

% ====================
% Body
% ====================

\begin{document}

\begin{frame}[t]
\begin{columns}[t]
\separatorcolumn

\begin{column}{\colwidth}

  \begin{block}{The longest chain protocol}
    
    \begin{figure}
      \centering
      \subcaptionbox{The longest chain protocol\label{fig:longest-chain}}{\tikzstyle{block} = [draw, fill=blue!20, rectangle, rounded corners, minimum height=2em, minimum width=3.24em, text centered, very thick]
\tikzstyle{fork} = [draw, dashed, fill=blue!20, rectangle, rounded corners, minimum height=2em, minimum width=3.24em, text centered, very thick]

\begin{tikzpicture}[auto, node distance=4em,>=latex']
\node [block,label=left:genesis] (b0) {};
\node [block, below of=b0] (b1) {};
\node [block, below of=b1] (b2) {};
\node [fork, right of=b2] (f2) {};
\node [block, below of=b2] (b3) {};
\node [fork, right of=b3] (f3) {};
\node [block, below of=b3, label=right:longest chain] (b4) {};


\draw[very thick, dashed] (b1) to (b0);
\draw[very thick, ->] (b2) to (b1);
\draw[very thick, ->] (b3) to (b2);
\draw[very thick, ->] (b4) to (b3);
\draw[very thick, ->] (f2) to (b1);
\draw[very thick, ->] (f3) to (f2);

\draw[decorate,decoration={brace,mirror,amplitude=15pt,raise=10pt},very thick] (b2.north west) -- (b4.south west) node [black,midway,xshift=-4.2cm] {$k$-deep};
\draw[decorate,decoration={brace,amplitude=15pt,raise=10pt},very thick] (f2.north east) -- (f3.south east) node [black,midway,xshift=1cm] {fork};
\end{tikzpicture}}
      \subcaptionbox{The Prism protocol\label{fig:prism-chains}}{\tikzstyle{proposer} = [draw, fill=blue!20, rectangle, rounded corners, minimum height=1.5em, minimum width=2.427em, text centered, very thick]
\tikzstyle{voter} = [draw, fill=red!20, rectangle, rounded corners, minimum height=1.5em, minimum width=2.427em, text centered, very thick]
\tikzstyle{transaction} = [draw, fill=green!20, circle, rounded corners, minimum height=0.75em, minimum width=0.75em, text centered, very thick]
\tikzstyle{pfork} = [draw, dashed, fill=blue!20, rectangle, rounded corners, minimum height=1.5em, minimum width=2.427em, text centered, very thick]
\tikzstyle{vfork} = [draw, dashed, fill=red!20, rectangle, rounded corners, minimum height=1.5em, minimum width=2.427em, text centered, very thick]
\tikzstyle{placeholder} = [draw=none, minimum height=1.5em, minimum width=2.427em]

\begin{tikzpicture}[auto, node distance=3em,>=latex']
\node [proposer] (p0) {};
\node [placeholder, right of=p0] (ph4) {};
\node [proposer, below of=p0] (p1) {};
\node [proposer, below of=p1] (p2) {};
\node [proposer, below of=p2] (p3) {};
\node [pfork, right of=p2, label=below:fork] (pf2) {};
\begin{scope}[node distance=1.2cm]
    \node[transaction, left of=p2, xshift=-2.5cm] (t1) {};
    \node[transaction, left of=t1] (t2) {};
    \node[transaction, below of=t1] (t3){};
    \node[transaction, below of=t2] (t4){};
    \node[transaction, above of=t1] (t5){};
    \node[transaction, above of=t2] (t6){};
\end{scope}
\node [voter, right of=ph4] (vl0) {};
\node [voter, below of=vl0] (vl1) {};
\node [voter, below of=vl1] (vl2) {};
\node [voter, below of=vl2] (vl3) {};
\node [placeholder, right of=vl0] (ph3) {};
\node [right of=vl1] {$\cdots$};
\node [voter, right of=ph3] (vr0) {};
\node [voter, below of=vr0] (vr1) {};
\node [voter, below of=vr1] (vr2) {};
\node [voter, below of=vr2] (vr3) {};
\node [vfork, right of=vr2, label=below:fork] (vf2) {};
\draw[very thick, dashed] (vl1) to (vl0);
\draw[very thick, ->] (vl2) to (vl1);
\draw[very thick, ->] (vl3) to (vl2);
\draw[very thick, dashed] (vr1) to (vr0);
\draw[very thick, ->] (vr2) to (vr1);
\draw[very thick, ->] (vr3) to (vr2);
\draw[very thick, ->] (vf2) to (vr1);
\draw[very thick, <-] (p2)  to [looseness=0.4] (vl2);
\draw[very thick, <-] (p2)  to [looseness=0.4] (vr2);
\draw[densely dashed, very thick, <-] (pf2)  to [looseness=0.4] (vf2);

\draw[very thick, <-] (t1)  to [looseness=0.5] (p2.west);
\draw[very thick, <-] (t2)  to [looseness=0.5] (p2.west);
\draw[very thick, <-] (t3)  to [looseness=0.3] (p2.south west);
\draw[very thick, <-] (t4)  to [looseness=0.3] (p2.south west);
\draw[very thick, <-] (t5)  to [looseness=0.8] (p2.north west);
\draw[very thick, <-] (t6)  to [looseness=0.8] (p2.north west);

\draw[decorate,decoration={brace,amplitude=15pt,mirror,raise=10pt},very thick] (vl3.south west) -- (vr3.south east) node [black,midway,yshift=-2cm] {voter};
\draw[decorate,decoration={brace,amplitude=15pt,mirror,raise=10pt},very thick] (p3.south west) -- (p3.south east) node [black,midway,yshift=-2cm] {proposer};
\draw[decorate,decoration={brace,amplitude=15pt,mirror,raise=15pt},very thick] (t4.south west) -- (t3.south east) node [black,midway,yshift=-2cm] {transaction};

\end{tikzpicture}}
      \caption{The longest chain protocol and the Prism protocol.}
    \end{figure}
      
  \heading{The Nakamoto longest chain protocol (Fig.~\ref{fig:longest-chain})}
  
  First proposed in the Bitcoin paper~\cite{nakamoto2008bitcoin} by Nakamoto.
  \begin{itemize}
    \item Blocks \textit{``mined''} by solving an inequality and \textit{``chained''} together by including a hash to the parent
    \item New blocks always append to the tip of the longest chain
    \item Blocks on the longest chain form an order (the \textit{ledger}) and everybody agrees on it
  \end{itemize}
  
  \heading{The \textit{double-spending} attack}
  
  \begin{enumerate}
      \item \textbf{Alice} sends \textbf{Bob} \$100 for a car. The transaction is recorded in the public longest chain.
      \item Meanwhile \textbf{Alice} starts to mine a \textbf{private fork} that does not contain this transaction.
      \item \textbf{Bob} observes the transaction and gives the car to \textbf{Alice}.
      \item \textbf{Alice} waits until the \textbf{private fork} is longer than the public longest chain, and releases it.
      \item The \textbf{private fork} becomes the public longest chain. \textbf{Bob} loses the money.
  \end{enumerate}
  
  \textbf{Solution}: \textit{Confirm} a block only when it is \textbf{$k$-deep} in the longest chain. As long as the adversary controls less than 50\% power, the confirmed block is very likely to stay in the longest chain forever. 
  
  \heading{Performance limitations}
    
  \textbf{Confirmation Latency}. Requires $k$-deep to defend against double-spending attack. $k$ is large if we want high security (Fig.~\ref{fig:latency}).
  
  \textbf{Throughput}. Honest blocks need to form a chain. Increasing the throughput causes the honest blocks to form a tree because they can not hear each other quickly enough. No longer safe against a 50\% adversary (Fig.~\ref{fig:throughput}, assuming every block travels 5 hops to reach the whole network).
  
  \begin{figure}
      \centering
      \subcaptionbox{Latency\label{fig:latency}}{\begin{tikzpicture}[gnuplot]
%% generated with GNUPLOT 5.2p7 (Lua 5.3; terminal rev. Nov 2018, script rev. 107)
%% Fri Oct 11 15:30:17 2019
\tikzset{every node/.append style={font={\fontsize{24.0pt}{28.8pt}\selectfont}}}
\path (0.000,0.000) rectangle (16.000,9.800);
\gpcolor{color=gp lt color border}
\gpsetlinetype{gp lt border}
\gpsetdashtype{gp dt solid}
\gpsetlinewidth{2.50}
\draw[gp path] (3.612,1.995)--(3.792,1.995);
\node[gp node right] at (3.170,1.995) {$10^{-6}$};
\draw[gp path] (3.612,4.350)--(3.792,4.350);
\node[gp node right] at (3.170,4.350) {$10^{-4}$};
\draw[gp path] (3.612,6.705)--(3.792,6.705);
\node[gp node right] at (3.170,6.705) {$10^{-2}$};
\draw[gp path] (3.612,9.060)--(3.792,9.060);
\node[gp node right] at (3.170,9.060) {$10^0$};
\draw[gp path] (4.054,1.995)--(4.054,2.175);
\node[gp node center] at (4.054,1.256) {$2$};
\draw[gp path] (5.824,1.995)--(5.824,2.175);
\node[gp node center] at (5.824,1.256) {$10$};
\draw[gp path] (8.036,1.995)--(8.036,2.175);
\node[gp node center] at (8.036,1.256) {$20$};
\draw[gp path] (10.249,1.995)--(10.249,2.175);
\node[gp node center] at (10.249,1.256) {$30$};
\draw[gp path] (12.461,1.995)--(12.461,2.175);
\node[gp node center] at (12.461,1.256) {$40$};
\draw[gp path] (14.673,1.995)--(14.673,2.175);
\node[gp node center] at (14.673,1.256) {$50$};
\draw[gp path] (3.612,9.060)--(3.612,1.995)--(14.673,1.995);
\node[gp node center,rotate=-270] at (0.591,5.527) {Attack Possibility};
\node[gp node center] at (9.142,0.516) {$k$};
\node[gp node right] at (11.399,8.510) {LC};
\gpcolor{rgb color={0.000,0.620,0.451}}
\gpsetlinewidth{3.50}
\draw[gp path] (11.841,8.510)--(13.789,8.510);
\draw[gp path] (3.612,9.060)--(14.673,1.995);
\gpcolor{color=gp lt color border}
\node[gp node right] at (11.399,7.771) {Prism};
\gpcolor{rgb color={0.898,0.118,0.063}}
\draw[gp path] (11.841,7.771)--(13.789,7.771);
\draw[gp path] (3.612,9.060)--(4.054,1.995);
\gpcolor{color=gp lt color border}
\gpsetlinewidth{2.50}
\draw[gp path] (3.612,9.060)--(3.612,1.995)--(14.673,1.995);
%% coordinates of the plot area
\gpdefrectangularnode{gp plot 1}{\pgfpoint{3.612cm}{1.995cm}}{\pgfpoint{14.673cm}{9.060cm}}
\end{tikzpicture}
%% gnuplot variables
}
      \subcaptionbox{Throughput\label{fig:throughput}}{\begin{tikzpicture}[gnuplot]
%% generated with GNUPLOT 5.2p7 (Lua 5.3; terminal rev. Nov 2018, script rev. 107)
%% Fri Oct 11 15:31:28 2019
\tikzset{every node/.append style={font={\fontsize{24.0pt}{28.8pt}\selectfont}}}
\path (0.000,0.000) rectangle (16.000,9.800);
\gpcolor{color=gp lt color border}
\gpsetlinetype{gp lt border}
\gpsetdashtype{gp dt solid}
\gpsetlinewidth{2.50}
\draw[gp path] (3.612,1.995)--(3.792,1.995);
\node[gp node right] at (3.170,1.995) {$0\%$};
\draw[gp path] (3.612,5.528)--(3.792,5.528);
\node[gp node right] at (3.170,5.528) {$25\%$};
\draw[gp path] (3.612,9.060)--(3.792,9.060);
\node[gp node right] at (3.170,9.060) {$50\%$};
\draw[gp path] (3.612,1.995)--(3.612,2.175);
\node[gp node center] at (3.612,1.256) {$10^{-3}$};
\draw[gp path] (7.299,1.995)--(7.299,2.175);
\node[gp node center] at (7.299,1.256) {$10^{-2}$};
\draw[gp path] (10.986,1.995)--(10.986,2.175);
\node[gp node center] at (10.986,1.256) {$10^{-1}$};
\draw[gp path] (14.673,1.995)--(14.673,2.175);
\node[gp node center] at (14.673,1.256) {$10^{0}$};
\draw[gp path] (3.612,9.060)--(3.612,1.995)--(14.673,1.995);
\node[gp node center,rotate=-270] at (0.591,5.527) {Adv. Fraction};
\node[gp node center] at (9.142,0.516) {Bandwidth Utilization};
\node[gp node right] at (6.264,3.283) {LC};
\gpcolor{rgb color={0.000,0.620,0.451}}
\gpsetlinewidth{3.50}
\draw[gp path] (6.706,3.283)--(8.654,3.283);
\draw[gp path] (3.612,9.042)--(3.724,9.041)--(3.835,9.040)--(3.947,9.038)--(4.059,9.037)%
  --(4.171,9.035)--(4.282,9.033)--(4.394,9.031)--(4.506,9.029)--(4.618,9.027)--(4.729,9.025)%
  --(4.841,9.022)--(4.953,9.019)--(5.064,9.017)--(5.176,9.013)--(5.288,9.010)--(5.400,9.006)%
  --(5.511,9.003)--(5.623,8.999)--(5.735,8.994)--(5.847,8.989)--(5.958,8.984)--(6.070,8.979)%
  --(6.182,8.973)--(6.293,8.967)--(6.405,8.960)--(6.517,8.953)--(6.629,8.946)--(6.740,8.938)%
  --(6.852,8.929)--(6.964,8.920)--(7.076,8.910)--(7.187,8.899)--(7.299,8.888)--(7.411,8.876)%
  --(7.522,8.863)--(7.634,8.849)--(7.746,8.834)--(7.858,8.818)--(7.969,8.801)--(8.081,8.783)%
  --(8.193,8.764)--(8.305,8.744)--(8.416,8.722)--(8.528,8.699)--(8.640,8.674)--(8.751,8.648)%
  --(8.863,8.620)--(8.975,8.590)--(9.087,8.559)--(9.198,8.525)--(9.310,8.490)--(9.422,8.452)%
  --(9.534,8.412)--(9.645,8.370)--(9.757,8.325)--(9.869,8.278)--(9.980,8.228)--(10.092,8.176)%
  --(10.204,8.120)--(10.316,8.062)--(10.427,8.001)--(10.539,7.936)--(10.651,7.869)--(10.763,7.798)%
  --(10.874,7.724)--(10.986,7.647)--(11.098,7.566)--(11.209,7.483)--(11.321,7.396)--(11.433,7.305)%
  --(11.545,7.211)--(11.656,7.115)--(11.768,7.015)--(11.880,6.912)--(11.992,6.806)--(12.103,6.698)%
  --(12.215,6.587)--(12.327,6.474)--(12.438,6.358)--(12.550,6.241)--(12.662,6.122)--(12.774,6.001)%
  --(12.885,5.880)--(12.997,5.757)--(13.109,5.634)--(13.221,5.511)--(13.332,5.388)--(13.444,5.265)%
  --(13.556,5.143)--(13.667,5.022)--(13.779,4.902)--(13.891,4.783)--(14.003,4.666)--(14.114,4.551)%
  --(14.226,4.439)--(14.338,4.328)--(14.450,4.221)--(14.561,4.116)--(14.673,4.014);
\gpcolor{color=gp lt color border}
\node[gp node right] at (6.264,2.544) {Prism};
\gpcolor{rgb color={0.898,0.118,0.063}}
\draw[gp path] (6.706,2.544)--(8.654,2.544);
\draw[gp path] (3.612,9.060)--(3.724,9.060)--(3.835,9.060)--(3.947,9.060)--(4.059,9.060)%
  --(4.171,9.060)--(4.282,9.060)--(4.394,9.060)--(4.506,9.060)--(4.618,9.060)--(4.729,9.060)%
  --(4.841,9.060)--(4.953,9.060)--(5.064,9.060)--(5.176,9.060)--(5.288,9.060)--(5.400,9.060)%
  --(5.511,9.060)--(5.623,9.060)--(5.735,9.060)--(5.847,9.060)--(5.958,9.060)--(6.070,9.060)%
  --(6.182,9.060)--(6.293,9.060)--(6.405,9.060)--(6.517,9.060)--(6.629,9.060)--(6.740,9.060)%
  --(6.852,9.060)--(6.964,9.060)--(7.076,9.060)--(7.187,9.060)--(7.299,9.060)--(7.411,9.060)%
  --(7.522,9.060)--(7.634,9.060)--(7.746,9.060)--(7.858,9.060)--(7.969,9.060)--(8.081,9.060)%
  --(8.193,9.060)--(8.305,9.060)--(8.416,9.060)--(8.528,9.060)--(8.640,9.060)--(8.751,9.060)%
  --(8.863,9.060)--(8.975,9.060)--(9.087,9.060)--(9.198,9.060)--(9.310,9.060)--(9.422,9.060)%
  --(9.534,9.060)--(9.645,9.060)--(9.757,9.060)--(9.869,9.060)--(9.980,9.060)--(10.092,9.060)%
  --(10.204,9.060)--(10.316,9.060)--(10.427,9.060)--(10.539,9.060)--(10.651,9.060)--(10.763,9.060)%
  --(10.874,9.060)--(10.986,9.060)--(11.098,9.060)--(11.209,9.060)--(11.321,9.060)--(11.433,9.060)%
  --(11.545,9.060)--(11.656,9.060)--(11.768,9.060)--(11.880,9.060)--(11.992,9.060)--(12.103,9.060)%
  --(12.215,9.060)--(12.327,9.060)--(12.438,9.060)--(12.550,9.060)--(12.662,9.060)--(12.774,9.060)%
  --(12.885,9.060)--(12.997,9.060)--(13.109,9.060)--(13.221,9.060)--(13.332,9.060)--(13.444,9.060)%
  --(13.556,9.060)--(13.667,9.060)--(13.779,9.060)--(13.891,9.060)--(14.003,9.060)--(14.114,9.060)%
  --(14.226,9.060)--(14.338,9.060)--(14.450,9.060)--(14.561,9.060)--(14.673,9.060);
\gpcolor{color=gp lt color border}
\gpsetlinewidth{2.50}
\draw[gp path] (3.612,9.060)--(3.612,1.995)--(14.673,1.995);
%% coordinates of the plot area
\gpdefrectangularnode{gp plot 1}{\pgfpoint{3.612cm}{1.995cm}}{\pgfpoint{14.673cm}{9.060cm}}
\end{tikzpicture}
%% gnuplot variables
}
      \caption{Performance limitation of the longest chain protocol (LC) and comparison with Prism.}
  \end{figure}
  \end{block}
  
\end{column}

\separatorcolumn

\begin{column}{\colwidth}

  
  \begin{alertblock}{The Prism protocol}

  \heading{Deconstruct the longest chain protocol}
  
  

  \end{alertblock}

\end{column}

\separatorcolumn

\begin{column}{\colwidth}

\begin{block}{Performance results}
  
    \begin{figure}
      \centering
      \begin{tikzpicture}[gnuplot]
%% generated with GNUPLOT 5.2p7 (Lua 5.3; terminal rev. Nov 2018, script rev. 107)
%% Thu Oct 10 16:50:07 2019
\tikzset{every node/.append style={font={\fontsize{24.0pt}{28.8pt}\selectfont}}}
\path (0.000,0.000) rectangle (26.000,16.000);
\gpcolor{color=gp lt color border}
\gpsetlinetype{gp lt border}
\gpsetdashtype{gp dt solid}
\gpsetlinewidth{2.50}
\draw[gp path] (3.170,1.995)--(3.350,1.995);
\draw[gp path] (24.673,1.995)--(24.493,1.995);
\node[gp node right] at (2.728,1.995) {$10^0$};
\draw[gp path] (3.170,2.794)--(3.260,2.794);
\draw[gp path] (24.673,2.794)--(24.583,2.794);
\draw[gp path] (3.170,3.261)--(3.260,3.261);
\draw[gp path] (24.673,3.261)--(24.583,3.261);
\draw[gp path] (3.170,3.592)--(3.260,3.592);
\draw[gp path] (24.673,3.592)--(24.583,3.592);
\draw[gp path] (3.170,3.849)--(3.260,3.849);
\draw[gp path] (24.673,3.849)--(24.583,3.849);
\draw[gp path] (3.170,4.059)--(3.260,4.059);
\draw[gp path] (24.673,4.059)--(24.583,4.059);
\draw[gp path] (3.170,4.237)--(3.260,4.237);
\draw[gp path] (24.673,4.237)--(24.583,4.237);
\draw[gp path] (3.170,4.391)--(3.260,4.391);
\draw[gp path] (24.673,4.391)--(24.583,4.391);
\draw[gp path] (3.170,4.527)--(3.260,4.527);
\draw[gp path] (24.673,4.527)--(24.583,4.527);
\draw[gp path] (3.170,4.648)--(3.350,4.648);
\draw[gp path] (24.673,4.648)--(24.493,4.648);
\node[gp node right] at (2.728,4.648) {$10^1$};
\draw[gp path] (3.170,5.447)--(3.260,5.447);
\draw[gp path] (24.673,5.447)--(24.583,5.447);
\draw[gp path] (3.170,5.914)--(3.260,5.914);
\draw[gp path] (24.673,5.914)--(24.583,5.914);
\draw[gp path] (3.170,6.245)--(3.260,6.245);
\draw[gp path] (24.673,6.245)--(24.583,6.245);
\draw[gp path] (3.170,6.502)--(3.260,6.502);
\draw[gp path] (24.673,6.502)--(24.583,6.502);
\draw[gp path] (3.170,6.712)--(3.260,6.712);
\draw[gp path] (24.673,6.712)--(24.583,6.712);
\draw[gp path] (3.170,6.890)--(3.260,6.890);
\draw[gp path] (24.673,6.890)--(24.583,6.890);
\draw[gp path] (3.170,7.044)--(3.260,7.044);
\draw[gp path] (24.673,7.044)--(24.583,7.044);
\draw[gp path] (3.170,7.180)--(3.260,7.180);
\draw[gp path] (24.673,7.180)--(24.583,7.180);
\draw[gp path] (3.170,7.301)--(3.350,7.301);
\draw[gp path] (24.673,7.301)--(24.493,7.301);
\node[gp node right] at (2.728,7.301) {$10^2$};
\draw[gp path] (3.170,8.100)--(3.260,8.100);
\draw[gp path] (24.673,8.100)--(24.583,8.100);
\draw[gp path] (3.170,8.567)--(3.260,8.567);
\draw[gp path] (24.673,8.567)--(24.583,8.567);
\draw[gp path] (3.170,8.898)--(3.260,8.898);
\draw[gp path] (24.673,8.898)--(24.583,8.898);
\draw[gp path] (3.170,9.155)--(3.260,9.155);
\draw[gp path] (24.673,9.155)--(24.583,9.155);
\draw[gp path] (3.170,9.365)--(3.260,9.365);
\draw[gp path] (24.673,9.365)--(24.583,9.365);
\draw[gp path] (3.170,9.543)--(3.260,9.543);
\draw[gp path] (24.673,9.543)--(24.583,9.543);
\draw[gp path] (3.170,9.697)--(3.260,9.697);
\draw[gp path] (24.673,9.697)--(24.583,9.697);
\draw[gp path] (3.170,9.833)--(3.260,9.833);
\draw[gp path] (24.673,9.833)--(24.583,9.833);
\draw[gp path] (3.170,9.954)--(3.350,9.954);
\draw[gp path] (24.673,9.954)--(24.493,9.954);
\node[gp node right] at (2.728,9.954) {$10^3$};
\draw[gp path] (3.170,10.753)--(3.260,10.753);
\draw[gp path] (24.673,10.753)--(24.583,10.753);
\draw[gp path] (3.170,11.220)--(3.260,11.220);
\draw[gp path] (24.673,11.220)--(24.583,11.220);
\draw[gp path] (3.170,11.551)--(3.260,11.551);
\draw[gp path] (24.673,11.551)--(24.583,11.551);
\draw[gp path] (3.170,11.808)--(3.260,11.808);
\draw[gp path] (24.673,11.808)--(24.583,11.808);
\draw[gp path] (3.170,12.018)--(3.260,12.018);
\draw[gp path] (24.673,12.018)--(24.583,12.018);
\draw[gp path] (3.170,12.196)--(3.260,12.196);
\draw[gp path] (24.673,12.196)--(24.583,12.196);
\draw[gp path] (3.170,12.350)--(3.260,12.350);
\draw[gp path] (24.673,12.350)--(24.583,12.350);
\draw[gp path] (3.170,12.486)--(3.260,12.486);
\draw[gp path] (24.673,12.486)--(24.583,12.486);
\draw[gp path] (3.170,12.607)--(3.350,12.607);
\draw[gp path] (24.673,12.607)--(24.493,12.607);
\node[gp node right] at (2.728,12.607) {$10^4$};
\draw[gp path] (3.170,13.406)--(3.260,13.406);
\draw[gp path] (24.673,13.406)--(24.583,13.406);
\draw[gp path] (3.170,13.873)--(3.260,13.873);
\draw[gp path] (24.673,13.873)--(24.583,13.873);
\draw[gp path] (3.170,14.204)--(3.260,14.204);
\draw[gp path] (24.673,14.204)--(24.583,14.204);
\draw[gp path] (3.170,14.461)--(3.260,14.461);
\draw[gp path] (24.673,14.461)--(24.583,14.461);
\draw[gp path] (3.170,14.671)--(3.260,14.671);
\draw[gp path] (24.673,14.671)--(24.583,14.671);
\draw[gp path] (3.170,14.849)--(3.260,14.849);
\draw[gp path] (24.673,14.849)--(24.583,14.849);
\draw[gp path] (3.170,15.003)--(3.260,15.003);
\draw[gp path] (24.673,15.003)--(24.583,15.003);
\draw[gp path] (3.170,15.139)--(3.260,15.139);
\draw[gp path] (24.673,15.139)--(24.583,15.139);
\draw[gp path] (3.170,15.260)--(3.350,15.260);
\draw[gp path] (24.673,15.260)--(24.493,15.260);
\node[gp node right] at (2.728,15.260) {$10^5$};
\draw[gp path] (3.170,1.995)--(3.170,2.175);
\draw[gp path] (3.170,15.260)--(3.170,15.080);
\node[gp node center] at (3.170,1.256) {$1$};
\draw[gp path] (5.137,1.995)--(5.137,2.085);
\draw[gp path] (5.137,15.260)--(5.137,15.170);
\draw[gp path] (6.287,1.995)--(6.287,2.085);
\draw[gp path] (6.287,15.260)--(6.287,15.170);
\draw[gp path] (7.103,1.995)--(7.103,2.085);
\draw[gp path] (7.103,15.260)--(7.103,15.170);
\draw[gp path] (7.737,1.995)--(7.737,2.085);
\draw[gp path] (7.737,15.260)--(7.737,15.170);
\draw[gp path] (8.254,1.995)--(8.254,2.085);
\draw[gp path] (8.254,15.260)--(8.254,15.170);
\draw[gp path] (8.691,1.995)--(8.691,2.085);
\draw[gp path] (8.691,15.260)--(8.691,15.170);
\draw[gp path] (9.070,1.995)--(9.070,2.085);
\draw[gp path] (9.070,15.260)--(9.070,15.170);
\draw[gp path] (9.404,1.995)--(9.404,2.085);
\draw[gp path] (9.404,15.260)--(9.404,15.170);
\draw[gp path] (9.703,1.995)--(9.703,2.175);
\draw[gp path] (9.703,15.260)--(9.703,15.080);
\node[gp node center] at (9.703,1.256) {$10$};
\draw[gp path] (11.670,1.995)--(11.670,2.085);
\draw[gp path] (11.670,15.260)--(11.670,15.170);
\draw[gp path] (12.820,1.995)--(12.820,2.085);
\draw[gp path] (12.820,15.260)--(12.820,15.170);
\draw[gp path] (13.637,1.995)--(13.637,2.085);
\draw[gp path] (13.637,15.260)--(13.637,15.170);
\draw[gp path] (14.270,1.995)--(14.270,2.085);
\draw[gp path] (14.270,15.260)--(14.270,15.170);
\draw[gp path] (14.787,1.995)--(14.787,2.085);
\draw[gp path] (14.787,15.260)--(14.787,15.170);
\draw[gp path] (15.224,1.995)--(15.224,2.085);
\draw[gp path] (15.224,15.260)--(15.224,15.170);
\draw[gp path] (15.603,1.995)--(15.603,2.085);
\draw[gp path] (15.603,15.260)--(15.603,15.170);
\draw[gp path] (15.938,1.995)--(15.938,2.085);
\draw[gp path] (15.938,15.260)--(15.938,15.170);
\draw[gp path] (16.236,1.995)--(16.236,2.175);
\draw[gp path] (16.236,15.260)--(16.236,15.080);
\node[gp node center] at (16.236,1.256) {$100$};
\draw[gp path] (18.203,1.995)--(18.203,2.085);
\draw[gp path] (18.203,15.260)--(18.203,15.170);
\draw[gp path] (19.354,1.995)--(19.354,2.085);
\draw[gp path] (19.354,15.260)--(19.354,15.170);
\draw[gp path] (20.170,1.995)--(20.170,2.085);
\draw[gp path] (20.170,15.260)--(20.170,15.170);
\draw[gp path] (20.803,1.995)--(20.803,2.085);
\draw[gp path] (20.803,15.260)--(20.803,15.170);
\draw[gp path] (21.320,1.995)--(21.320,2.085);
\draw[gp path] (21.320,15.260)--(21.320,15.170);
\draw[gp path] (21.758,1.995)--(21.758,2.085);
\draw[gp path] (21.758,15.260)--(21.758,15.170);
\draw[gp path] (22.137,1.995)--(22.137,2.085);
\draw[gp path] (22.137,15.260)--(22.137,15.170);
\draw[gp path] (22.471,1.995)--(22.471,2.085);
\draw[gp path] (22.471,15.260)--(22.471,15.170);
\draw[gp path] (22.770,1.995)--(22.770,2.175);
\draw[gp path] (22.770,15.260)--(22.770,15.080);
\node[gp node center] at (22.770,1.256) {$1000$};
\draw[gp path] (3.170,15.260)--(3.170,1.995)--(24.673,1.995)--(24.673,15.260)--cycle;
\node[gp node right] at (8.254,10.421) {$\beta=20\%$};
\node[gp node right] at (17.742,7.301) {$\beta=20\%$};
\node[gp node right] at (22.207,4.648) {$\beta=33\%$};
\node[gp node center,rotate=-270] at (0.591,8.627) {Throughput (tps)};
\node[gp node center] at (13.921,0.516) {Latency (s)};
\gpsetlinewidth{3.00}
\draw[gp path] (3.612,2.175)--(3.612,4.392)--(12.190,4.392)--(12.190,2.175)--cycle;
\node[gp node right] at (9.358,4.022) {Algorand};
\gpcolor{rgb color={0.580,0.000,0.827}}
\gpsetlinewidth{3.50}
\draw[gp path] (9.800,4.022)--(11.748,4.022);
\draw[gp path] (5.599,9.258)--(7.834,9.758)--(8.973,10.095)--(10.181,10.071)--(10.171,10.070)%
  --(10.834,10.126)--(11.371,10.159)--(13.415,10.109)--(14.326,10.227)--(15.014,10.270)--(15.235,10.185)%
  --(15.650,10.226)--(15.979,10.235)--(16.348,10.244);
\gpsetpointsize{16.00}
\gppoint{gp mark 1}{(5.599,9.258)}
\gppoint{gp mark 1}{(7.834,9.758)}
\gppoint{gp mark 1}{(8.973,10.095)}
\gppoint{gp mark 1}{(10.181,10.071)}
\gppoint{gp mark 1}{(10.171,10.070)}
\gppoint{gp mark 1}{(10.834,10.126)}
\gppoint{gp mark 1}{(11.371,10.159)}
\gppoint{gp mark 1}{(13.415,10.109)}
\gppoint{gp mark 1}{(14.326,10.227)}
\gppoint{gp mark 1}{(15.014,10.270)}
\gppoint{gp mark 1}{(15.235,10.185)}
\gppoint{gp mark 1}{(15.650,10.226)}
\gppoint{gp mark 1}{(15.979,10.235)}
\gppoint{gp mark 1}{(16.348,10.244)}
\gppoint{gp mark 1}{(10.774,4.022)}
\gpcolor{color=gp lt color border}
\node[gp node right] at (9.358,3.283) {Longest Chain};
\gpcolor{rgb color={0.000,0.620,0.451}}
\draw[gp path] (9.800,3.283)--(11.748,3.283);
\draw[gp path] (16.445,3.623)--(17.765,6.977)--(18.094,8.333)--(19.882,9.422)--(20.404,9.864)%
  --(21.018,10.680)--(22.394,11.505)--(24.384,12.149);
\gppoint{gp mark 2}{(16.445,3.623)}
\gppoint{gp mark 2}{(17.765,6.977)}
\gppoint{gp mark 2}{(18.094,8.333)}
\gppoint{gp mark 2}{(19.882,9.422)}
\gppoint{gp mark 2}{(20.404,9.864)}
\gppoint{gp mark 2}{(21.018,10.680)}
\gppoint{gp mark 2}{(22.394,11.505)}
\gppoint{gp mark 2}{(24.384,12.149)}
\gppoint{gp mark 2}{(10.774,3.283)}
\draw[gp path] (21.500,2.591)--(22.766,5.992)--(22.956,7.498)--(24.673,8.497);
\gppoint{gp mark 2}{(21.500,2.591)}
\gppoint{gp mark 2}{(22.766,5.992)}
\gppoint{gp mark 2}{(22.956,7.498)}
\gppoint{gp mark 2}{(24.673,8.497)}
\gpcolor{color=gp lt color border}
\node[gp node right] at (9.358,2.544) {Prism};
\node[gp node right,rotate=45] at (10.242,14.493) {$\beta=20\%$};
\gpcolor{rgb color={0.898,0.118,0.063}}
\gpsetlinewidth{5.00}
\gpsetpointsize{20.00}
\gppoint{gp mark 1}{(10.461,14.920)}
\gpcolor{color=gp lt color border}
\node[gp node right,rotate=45] at (11.882,14.517) {$\beta=33\%$};
\gpcolor{rgb color={0.898,0.118,0.063}}
\gppoint{gp mark 1}{(12.101,14.944)}
\gpcolor{color=gp lt color border}
\node[gp node right,rotate=45] at (13.321,14.540) {$\beta=40\%$};
\gpcolor{rgb color={0.898,0.118,0.063}}
\gppoint{gp mark 1}{(13.540,14.967)}
\gpcolor{color=gp lt color border}
\node[gp node right,rotate=45] at (14.343,14.608) {$\beta=42\%$};
\gpcolor{rgb color={0.898,0.118,0.063}}
\gppoint{gp mark 1}{(14.562,15.035)}
\gpcolor{color=gp lt color border}
\node[gp node right,rotate=45] at (15.560,14.513) {$\beta=43\%$};
\gpcolor{rgb color={0.898,0.118,0.063}}
\gppoint{gp mark 1}{(15.779,14.940)}
\gpcolor{color=gp lt color border}
\node[gp node right,rotate=45] at (19.096,14.486) {$\beta=44\%$};
\gpcolor{rgb color={0.898,0.118,0.063}}
\gppoint{gp mark 1}{(19.315,14.913)}
\gppoint{gp mark 1}{(10.774,2.544)}
\gpcolor{color=gp lt color border}
\gpsetlinewidth{2.50}
\draw[gp path] (3.170,15.260)--(3.170,1.995)--(24.673,1.995)--(24.673,15.260)--cycle;
%% coordinates of the plot area
\gpdefrectangularnode{gp plot 1}{\pgfpoint{3.170cm}{1.995cm}}{\pgfpoint{24.673cm}{15.260cm}}
\end{tikzpicture}
%% gnuplot variables

      \caption{Performance of Prism, Algorand~\cite{algorand}, and the longest chain protocol. Curves represent tradeoff.}
    \end{figure}
  \end{block}

  \begin{block}{References}

    \nocite{*}
    \footnotesize{\bibliographystyle{plain}\bibliography{poster}}

  \end{block}

\end{column}

\separatorcolumn
\end{columns}
\end{frame}

\end{document}
