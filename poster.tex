% Gemini theme
% https://github.com/anishathalye/gemini

\documentclass[final]{beamer}

% ====================
% Packages
% ====================

\usepackage[T1]{fontenc}
\usepackage{lmodern}
\usepackage[size=custom,width=120,height=72,scale=1.0]{beamerposter}
\usetheme{gemini}
\usecolortheme{gemini}
\usepackage{subcaption}
\usepackage{tikz}
\usepackage{gnuplot-lua-tikz}
\usepackage{graphicx}
\usepackage{booktabs}
\usepackage{tikz}
\usetikzlibrary{positioning}
\usepackage{pgfplots}
\usepackage{amsmath}
\usepackage{multicol}
\usepackage{multirow}

% ====================
% Lengths
% ====================

% If you have N columns, choose \sepwidth and \colwidth such that
% (N+1)*\sepwidth + N*\colwidth = \paperwidth
\newlength{\sepwidth}
\newlength{\colwidth}
\setlength{\sepwidth}{0.025\paperwidth}
\setlength{\colwidth}{0.3\paperwidth}

\newcommand{\separatorcolumn}{\begin{column}{\sepwidth}\end{column}}

% ====================
% Title
% ====================

\title{Prism: Scaling Bitcoin by 10,000$\boldsymbol{\times}$}

\author{Lei Yang \inst{1} \and Vivek Bagaria \inst{2} \and Gerui Wang \inst{3} \and Mohammad Alizadeh \inst{1} \and David Tse \inst{2} \and Giulia Fanti \inst{4} \and Pramod Viswanath \inst{3}}

\institute[shortinst]{\inst{1} MIT CSAIL \samelineand \inst{2} Stanford University \samelineand \inst{3} University of Illinois at Urbana-Champaign \samelineand \inst{4} Carnegie Mellon University}

% ====================
% Body
% ====================

\begin{document}

\begin{frame}[t]
\begin{columns}[t]
\separatorcolumn

\begin{column}{\colwidth}

  \begin{block}{The longest chain protocol}
    
    \begin{figure}
      \centering
      \subcaptionbox{The longest chain protocol\label{fig:longest-chain}}{\input{longest-chain.tex}}
      \subcaptionbox{The Prism protocol\label{fig:prism-chains}}{\tikzstyle{block} = [draw, fill=blue!20, rectangle, rounded corners, minimum height=1.5em, minimum width=2.427em, text centered, very thick]
\tikzstyle{fork} = [draw, dashed, fill=blue!20, rectangle, rounded corners, minimum height=1.5em, minimum width=2.427em, text centered, very thick]

\begin{tikzpicture}[auto, node distance=3em,>=latex']
\node [block,label=left:genesis] (b0) {};
\node [block, below of=b0] (b1) {};
\node [block, below of=b1] (b2) {};
\node [fork, right of=b2] (f2) {};
\node [block, below of=b2] (b3) {};
\node [fork, right of=b3] (f3) {};
\node [block, below of=b3, label=right:longest chain] (b4) {};


\draw[very thick, dashed] (b1) to (b0);
\draw[very thick, ->] (b2) to (b1);
\draw[very thick, ->] (b3) to (b2);
\draw[very thick, ->] (b4) to (b3);
\draw[very thick, ->] (f2) to (b1);
\draw[very thick, ->] (f3) to (f2);

\draw[decorate,decoration={brace,mirror,amplitude=15pt,raise=10pt},very thick] (b2.north west) -- (b4.south west) node [black,midway,xshift=-4.2cm] {$k$-deep};
\draw[decorate,decoration={brace,amplitude=15pt,raise=10pt},very thick] (f2.north east) -- (f3.south east) node [black,midway,xshift=1cm] {fork};
\end{tikzpicture}}
      \caption{The longest chain protocol and the Prism protocol.}
    \end{figure}
      
  \heading{The Nakamoto longest chain protocol (Fig.~\ref{fig:longest-chain})}
  
  First proposed in the Bitcoin paper by Nakamoto.
  \begin{itemize}
    \item Blocks \textit{``mined''} by solving an inequality and \textit{``chained''} together by including a hash to the parent
    \item New blocks always append to the tip of the longest chain
    \item Blocks on the longest chain form an order (the \textit{ledger}) and everybody agrees on it
  \end{itemize}
  
  \heading{The \textit{double-spending} attack}
  
  \begin{enumerate}
      \item \textbf{Alice} sends \textbf{Bob} \$100 for a car. The transaction is recorded in the public longest chain.
      \item Meanwhile \textbf{Alice} starts to mine a \textbf{private fork} that does not contain this transaction.
      \item \textbf{Bob} observes the transaction and gives the car to \textbf{Alice}.
      \item \textbf{Alice} waits until the \textbf{private fork} is longer than the public longest chain, and releases it.
      \item The \textbf{private fork} becomes the public longest chain. \textbf{Bob} loses the money.
  \end{enumerate}
  
  \textbf{Solution}: \textit{Confirm} a block only when it is \textbf{$k$-deep} in the longest chain. As long as the adversary controls less than 50\% power, the confirmed block is very likely to stay in the longest chain forever. 
  
  \heading{Performance limitations}
    
  \textbf{Confirmation Latency}. Requires $k$-deep to defend against double-spending attack. $k$ is large if we want high security (Fig.~\ref{fig:latency}).
  
  \textbf{Throughput}. Honest blocks need to form a chain. Increasing the throughput causes the honest blocks to form a tree because they can not hear each other quickly enough. No longer safe against a 50\% adversary (Fig.~\ref{fig:throughput}, assuming every block travels 5 hops to reach the whole network).
  
  \begin{figure}
      \centering
      \subcaptionbox{Latency\label{fig:latency}}{\begin{tikzpicture}[gnuplot]
%% generated with GNUPLOT 5.2p7 (Lua 5.3; terminal rev. Nov 2018, script rev. 107)
%% Fri Oct 11 15:30:17 2019
\tikzset{every node/.append style={font={\fontsize{24.0pt}{28.8pt}\selectfont}}}
\path (0.000,0.000) rectangle (16.000,9.800);
\gpcolor{color=gp lt color border}
\gpsetlinetype{gp lt border}
\gpsetdashtype{gp dt solid}
\gpsetlinewidth{2.50}
\draw[gp path] (3.612,1.995)--(3.792,1.995);
\node[gp node right] at (3.170,1.995) {$10^{-6}$};
\draw[gp path] (3.612,4.350)--(3.792,4.350);
\node[gp node right] at (3.170,4.350) {$10^{-4}$};
\draw[gp path] (3.612,6.705)--(3.792,6.705);
\node[gp node right] at (3.170,6.705) {$10^{-2}$};
\draw[gp path] (3.612,9.060)--(3.792,9.060);
\node[gp node right] at (3.170,9.060) {$10^0$};
\draw[gp path] (4.054,1.995)--(4.054,2.175);
\node[gp node center] at (4.054,1.256) {$2$};
\draw[gp path] (5.824,1.995)--(5.824,2.175);
\node[gp node center] at (5.824,1.256) {$10$};
\draw[gp path] (8.036,1.995)--(8.036,2.175);
\node[gp node center] at (8.036,1.256) {$20$};
\draw[gp path] (10.249,1.995)--(10.249,2.175);
\node[gp node center] at (10.249,1.256) {$30$};
\draw[gp path] (12.461,1.995)--(12.461,2.175);
\node[gp node center] at (12.461,1.256) {$40$};
\draw[gp path] (14.673,1.995)--(14.673,2.175);
\node[gp node center] at (14.673,1.256) {$50$};
\draw[gp path] (3.612,9.060)--(3.612,1.995)--(14.673,1.995);
\node[gp node center,rotate=-270] at (0.591,5.527) {Attack Possibility};
\node[gp node center] at (9.142,0.516) {$k$};
\node[gp node right] at (11.399,8.510) {LC};
\gpcolor{rgb color={0.000,0.620,0.451}}
\gpsetlinewidth{3.50}
\draw[gp path] (11.841,8.510)--(13.789,8.510);
\draw[gp path] (3.612,9.060)--(14.673,1.995);
\gpcolor{color=gp lt color border}
\node[gp node right] at (11.399,7.771) {Prism};
\gpcolor{rgb color={0.898,0.118,0.063}}
\draw[gp path] (11.841,7.771)--(13.789,7.771);
\draw[gp path] (3.612,9.060)--(4.054,1.995);
\gpcolor{color=gp lt color border}
\gpsetlinewidth{2.50}
\draw[gp path] (3.612,9.060)--(3.612,1.995)--(14.673,1.995);
%% coordinates of the plot area
\gpdefrectangularnode{gp plot 1}{\pgfpoint{3.612cm}{1.995cm}}{\pgfpoint{14.673cm}{9.060cm}}
\end{tikzpicture}
%% gnuplot variables
}
      \subcaptionbox{Throughput\label{fig:throughput}}{\begin{tikzpicture}[gnuplot]
%% generated with GNUPLOT 5.2p7 (Lua 5.3; terminal rev. Nov 2018, script rev. 107)
%% Fri Oct 11 22:23:51 2019
\tikzset{every node/.append style={font={\fontsize{24.0pt}{28.8pt}\selectfont}}}
\path (0.000,0.000) rectangle (16.000,9.800);
\gpcolor{color=gp lt color border}
\gpsetlinetype{gp lt border}
\gpsetdashtype{gp dt solid}
\gpsetlinewidth{2.50}
\draw[gp path] (3.612,1.995)--(3.792,1.995);
\node[gp node right] at (3.170,1.995) {$0\%$};
\draw[gp path] (3.612,5.528)--(3.792,5.528);
\node[gp node right] at (3.170,5.528) {$25\%$};
\draw[gp path] (3.612,9.060)--(3.792,9.060);
\node[gp node right] at (3.170,9.060) {$50\%$};
\draw[gp path] (3.612,1.995)--(3.612,2.175);
\node[gp node center] at (3.612,1.256) {$10^{-3}$};
\draw[gp path] (7.299,1.995)--(7.299,2.175);
\node[gp node center] at (7.299,1.256) {$10^{-2}$};
\draw[gp path] (10.986,1.995)--(10.986,2.175);
\node[gp node center] at (10.986,1.256) {$10^{-1}$};
\draw[gp path] (14.673,1.995)--(14.673,2.175);
\node[gp node center] at (14.673,1.256) {$10^{0}$};
\draw[gp path] (3.612,9.060)--(3.612,1.995)--(14.673,1.995);
\node[gp node center,rotate=-270] at (0.591,5.527) {Adv. Fraction};
\node[gp node center] at (9.142,0.516) {Bandwidth Utilization};
\node[gp node right] at (6.264,3.283) {LC};
\gpcolor{rgb color={0.000,0.620,0.451}}
\gpsetlinewidth{3.50}
\draw[gp path] (6.706,3.283)--(8.654,3.283);
\draw[gp path] (3.612,9.042)--(3.724,9.041)--(3.835,9.040)--(3.947,9.038)--(4.059,9.037)%
  --(4.171,9.035)--(4.282,9.033)--(4.394,9.031)--(4.506,9.029)--(4.618,9.027)--(4.729,9.025)%
  --(4.841,9.022)--(4.953,9.019)--(5.064,9.017)--(5.176,9.013)--(5.288,9.010)--(5.400,9.006)%
  --(5.511,9.003)--(5.623,8.999)--(5.735,8.994)--(5.847,8.989)--(5.958,8.984)--(6.070,8.979)%
  --(6.182,8.973)--(6.293,8.967)--(6.405,8.960)--(6.517,8.953)--(6.629,8.946)--(6.740,8.938)%
  --(6.852,8.929)--(6.964,8.920)--(7.076,8.910)--(7.187,8.899)--(7.299,8.888)--(7.411,8.876)%
  --(7.522,8.863)--(7.634,8.849)--(7.746,8.834)--(7.858,8.818)--(7.969,8.801)--(8.081,8.783)%
  --(8.193,8.764)--(8.305,8.744)--(8.416,8.722)--(8.528,8.699)--(8.640,8.674)--(8.751,8.648)%
  --(8.863,8.620)--(8.975,8.590)--(9.087,8.559)--(9.198,8.525)--(9.310,8.490)--(9.422,8.452)%
  --(9.534,8.412)--(9.645,8.370)--(9.757,8.325)--(9.869,8.278)--(9.980,8.228)--(10.092,8.176)%
  --(10.204,8.120)--(10.316,8.062)--(10.427,8.001)--(10.539,7.936)--(10.651,7.869)--(10.763,7.798)%
  --(10.874,7.724)--(10.986,7.647)--(11.098,7.566)--(11.209,7.483)--(11.321,7.396)--(11.433,7.305)%
  --(11.545,7.211)--(11.656,7.115)--(11.768,7.015)--(11.880,6.912)--(11.992,6.806)--(12.103,6.698)%
  --(12.215,6.587)--(12.327,6.474)--(12.438,6.358)--(12.550,6.241)--(12.662,6.122)--(12.774,6.001)%
  --(12.885,5.880)--(12.997,5.757)--(13.109,5.634)--(13.221,5.511)--(13.332,5.388)--(13.444,5.265)%
  --(13.556,5.143)--(13.667,5.022)--(13.779,4.902)--(13.891,4.783)--(14.003,4.666)--(14.114,4.551)%
  --(14.226,4.439)--(14.338,4.328)--(14.450,4.221)--(14.561,4.116)--(14.673,4.014);
\gpcolor{color=gp lt color border}
\node[gp node right] at (6.264,2.544) {Prism};
\gpcolor{rgb color={0.898,0.118,0.063}}
\draw[gp path] (6.706,2.544)--(8.654,2.544);
\draw[gp path] (3.612,9.060)--(3.724,9.060)--(3.835,9.060)--(3.947,9.060)--(4.059,9.060)%
  --(4.171,9.060)--(4.282,9.060)--(4.394,9.060)--(4.506,9.060)--(4.618,9.060)--(4.729,9.060)%
  --(4.841,9.060)--(4.953,9.060)--(5.064,9.060)--(5.176,9.060)--(5.288,9.060)--(5.400,9.060)%
  --(5.511,9.060)--(5.623,9.060)--(5.735,9.060)--(5.847,9.060)--(5.958,9.060)--(6.070,9.060)%
  --(6.182,9.060)--(6.293,9.060)--(6.405,9.060)--(6.517,9.060)--(6.629,9.060)--(6.740,9.060)%
  --(6.852,9.060)--(6.964,9.060)--(7.076,9.060)--(7.187,9.060)--(7.299,9.060)--(7.411,9.060)%
  --(7.522,9.060)--(7.634,9.060)--(7.746,9.060)--(7.858,9.060)--(7.969,9.060)--(8.081,9.060)%
  --(8.193,9.060)--(8.305,9.060)--(8.416,9.060)--(8.528,9.060)--(8.640,9.060)--(8.751,9.060)%
  --(8.863,9.060)--(8.975,9.060)--(9.087,9.060)--(9.198,9.060)--(9.310,9.060)--(9.422,9.060)%
  --(9.534,9.060)--(9.645,9.060)--(9.757,9.060)--(9.869,9.060)--(9.980,9.060)--(10.092,9.060)%
  --(10.204,9.060)--(10.316,9.060)--(10.427,9.060)--(10.539,9.060)--(10.651,9.060)--(10.763,9.060)%
  --(10.874,9.060)--(10.986,9.060)--(11.098,9.060)--(11.209,9.060)--(11.321,9.060)--(11.433,9.060)%
  --(11.545,9.060)--(11.656,9.060)--(11.768,9.060)--(11.880,9.060)--(11.992,9.060)--(12.103,9.060)%
  --(12.215,9.060)--(12.327,9.060)--(12.438,9.060)--(12.550,9.060)--(12.662,9.060)--(12.774,9.060)%
  --(12.885,9.060)--(12.997,9.060)--(13.109,9.060)--(13.221,9.060)--(13.332,9.060)--(13.444,9.060)%
  --(13.556,9.060)--(13.667,9.060)--(13.779,9.060)--(13.891,9.060)--(14.003,9.060)--(14.114,9.060)%
  --(14.226,9.060)--(14.338,9.060)--(14.450,9.060)--(14.561,9.060)--(14.673,9.060);
\gpcolor{color=gp lt color border}
\gpsetlinewidth{2.50}
\draw[gp path] (3.612,9.060)--(3.612,1.995)--(14.673,1.995);
%% coordinates of the plot area
\gpdefrectangularnode{gp plot 1}{\pgfpoint{3.612cm}{1.995cm}}{\pgfpoint{14.673cm}{9.060cm}}
\end{tikzpicture}
%% gnuplot variables
}
      \caption{Performance limitation of the longest chain protocol (LC) and comparison with Prism.}
  \end{figure}
  \end{block}
  
\end{column}

\separatorcolumn

\begin{column}{\colwidth}

  
  \begin{alertblock}{The Prism protocol}

  \textbf{Observation}: blocks in the longest chain protocol serve three roles: standing for election to be leaders; adding transactions to the ledger; voting for ancestor blocks.
  
  \textbf{Solution}: factorizing the blocks according to those roles (Fig.~\ref{fig:prism-chains})
  
  \begin{enumerate}
      \item \textbf{Proposer blocktree} refers to \textit{transaction blocks}. One \textit{leader} is elected from each level. The sequence of the leaders decides the final order of the ledger.
      \item \textbf{Voter blockchains} each votes for one proposer block for every proposer level. Only votes on the respective longest chains are counted.  
      \item \textbf{Transaction blocks} contain transaction data. They do not form any specific topology.
  \end{enumerate}
  
  \begin{multicols}{2}
  \begin{align*}
   \text{\textbf{Latency}} &\leq D c_1(\beta) + \frac{D c_2(\beta)}{m}\log \frac{1}{\epsilon} \\
   \text{\textbf{Throughput}} &= (1-\beta) C
  \end{align*}
  \break 
  \begin{align*}
      D &= \text{network propagation delay} \\
      C &= \text{network bandwidth} \\
      m &= \text{number of voter chains} \\
      \epsilon &= \text{probability of double-spending attack} \\
      \beta &= \text{tolerable adversarial fraction} \\
      c_1\text{, }c_2 &= \beta\text{-dependent constants.} 
  \end{align*}
  \end{multicols}
  \end{alertblock}
  
  \begin{block}{Achieving high performance in the real world}
  
  \begin{figure}
      \centering
      \tikzstyle{ledger} = [draw, very thick, fill=blue!20, rectangle, 
    minimum height=4em, minimum width=6em, text centered, text width=5em]
\tikzstyle{blockchain} = [draw, very thick, fill=red!20, rectangle, minimum height=4em, minimum width=6em, text centered, text width=5em]
\tikzstyle{miner} = [draw, very thick, fill=green!20, rectangle, minimum height=4em, minimum width=6em, text centered, text width=5em]

\tikzstyle{database} = [draw, very thick, fill=yellow!20, rectangle, rounded corners, text width=5em, minimum height=3em, minimum width=6em, text centered]

% The block diagram code is probably more verbose than necessary
\begin{tikzpicture}[auto, node distance=1.8em and 3.3em,>=latex']
    \node [database] (blockchaindb) {Block Structure Database};
    \node [blockchain, above=of blockchaindb] (blockchain) {Block Structure Manager};
    \node [ledger, left=of blockchaindb] (ledger) {Ledger Manager};
    \node [miner, right=of blockchaindb] (miner) {Miner};
    \node [database, left=of ledger] (utxodb) {UTXO Database};
    \node [database, right=of blockchain] (mempool) {Memory Pool};
    \node [database, left=of blockchain] (blockdb) {Block Database};
    \node [above=of blockchain] (peers) {Peers};
    \node [above=of mempool] (newtx) {New Transactions};
    \draw [<->, very thick] (blockchaindb) -- node[name=a] {} (blockchain);
    \draw [->, very thick] (blockchaindb) -- node[name=b] {} (miner);
    \draw [<->, very thick] (blockchaindb) -- node[name=c] {} (ledger);
    \draw [<-, very thick] (blockchain) -- node[name=d] {} (miner);
    \draw [<->, very thick] (miner) -- node[name=e]{} (mempool);
    \draw [->, very thick] (blockchain) -- node[name=f]{} (mempool);
    \draw [<->, very thick] (blockchain) -- node[name=g]{} (blockdb);
    \draw [<->, very thick] (ledger) -- node[name=h]{} (utxodb);
    \draw [<-, very thick] (ledger) -- node[name=i]{} (blockdb);
    \draw [<->, very thick] (peers) -- node[name=j]{} (blockchain);
    \draw [->, very thick] (newtx) -- node[name=k]{} (mempool);


\end{tikzpicture}
      \caption{System design of the Prism client.}
        \end{figure}

      We implement the Prism consensus protocol and a UTXO-based cryptocurrency with pay-to-public-key transaction in 10,000 lines of Rust. Key design choices to enable high performance:
      
      \begin{itemize}
          \item \textbf{Asynchronous ledger updates}: Do not update the ledger every time a new block comes, so handling incoming blocks become parallelizable and we can handle hundreds to thousands of blocks per second as required by Prism.
          \item \textbf{Parallel transaction execution}: Execute transactions in parallel. We overcome data conflicts by borrowing \textit{pessimistic concurrency control} from database research. 	 
          \item \textbf{Functional-style design pattern}: Design each module such that the output only depends on the inputs. This eliminates cross-module shared states and results in a system that has no global lock.
          \item \textbf{No transaction broadcasting}: Prism mine blocks at very high rate. Each individual miner is very likely to mine a block in time comparable to delay associated with broadcasting a transaction (seconds).
      \end{itemize}
  \end{block}
    
  \textbf{Testbed}: 100 Amazon Web Service EC2 \texttt{c5d.4xlarge} instances. 16 CPU cores, 16 GB RAM, 400 GB NVMe SSD, 10 Gbps network. Instances connected into random 4-regular graph. 240 ms artificial RTT on each link. 400 Mbps bandwidth cap at each node.  
  
\end{column}

\separatorcolumn

\begin{column}{\colwidth}

\begin{block}{Evaluation}
  \heading{Performance}
    \begin{figure}
      \centering
      \begin{tikzpicture}[gnuplot]
%% generated with GNUPLOT 5.2p7 (Lua 5.3; terminal rev. Nov 2018, script rev. 107)
%% Thu Oct 10 15:03:55 2019
\tikzset{every node/.append style={font={\fontsize{20.0pt}{24.0pt}\selectfont}}}
\path (0.000,0.000) rectangle (22.000,13.600);
\gpcolor{color=gp lt color border}
\gpsetlinetype{gp lt border}
\gpsetdashtype{gp dt solid}
\gpsetlinewidth{1.00}
\draw[gp path] (2.640,1.663)--(2.820,1.663);
\draw[gp path] (20.895,1.663)--(20.715,1.663);
\node[gp node right] at (2.272,1.663) {$10^0$};
\draw[gp path] (2.640,2.345)--(2.730,2.345);
\draw[gp path] (20.895,2.345)--(20.805,2.345);
\draw[gp path] (2.640,2.743)--(2.730,2.743);
\draw[gp path] (20.895,2.743)--(20.805,2.743);
\draw[gp path] (2.640,3.026)--(2.730,3.026);
\draw[gp path] (20.895,3.026)--(20.805,3.026);
\draw[gp path] (2.640,3.245)--(2.730,3.245);
\draw[gp path] (20.895,3.245)--(20.805,3.245);
\draw[gp path] (2.640,3.425)--(2.730,3.425);
\draw[gp path] (20.895,3.425)--(20.805,3.425);
\draw[gp path] (2.640,3.576)--(2.730,3.576);
\draw[gp path] (20.895,3.576)--(20.805,3.576);
\draw[gp path] (2.640,3.708)--(2.730,3.708);
\draw[gp path] (20.895,3.708)--(20.805,3.708);
\draw[gp path] (2.640,3.823)--(2.730,3.823);
\draw[gp path] (20.895,3.823)--(20.805,3.823);
\draw[gp path] (2.640,3.927)--(2.820,3.927);
\draw[gp path] (20.895,3.927)--(20.715,3.927);
\node[gp node right] at (2.272,3.927) {$10^1$};
\draw[gp path] (2.640,4.609)--(2.730,4.609);
\draw[gp path] (20.895,4.609)--(20.805,4.609);
\draw[gp path] (2.640,5.007)--(2.730,5.007);
\draw[gp path] (20.895,5.007)--(20.805,5.007);
\draw[gp path] (2.640,5.290)--(2.730,5.290);
\draw[gp path] (20.895,5.290)--(20.805,5.290);
\draw[gp path] (2.640,5.509)--(2.730,5.509);
\draw[gp path] (20.895,5.509)--(20.805,5.509);
\draw[gp path] (2.640,5.689)--(2.730,5.689);
\draw[gp path] (20.895,5.689)--(20.805,5.689);
\draw[gp path] (2.640,5.840)--(2.730,5.840);
\draw[gp path] (20.895,5.840)--(20.805,5.840);
\draw[gp path] (2.640,5.972)--(2.730,5.972);
\draw[gp path] (20.895,5.972)--(20.805,5.972);
\draw[gp path] (2.640,6.087)--(2.730,6.087);
\draw[gp path] (20.895,6.087)--(20.805,6.087);
\draw[gp path] (2.640,6.191)--(2.820,6.191);
\draw[gp path] (20.895,6.191)--(20.715,6.191);
\node[gp node right] at (2.272,6.191) {$10^2$};
\draw[gp path] (2.640,6.873)--(2.730,6.873);
\draw[gp path] (20.895,6.873)--(20.805,6.873);
\draw[gp path] (2.640,7.271)--(2.730,7.271);
\draw[gp path] (20.895,7.271)--(20.805,7.271);
\draw[gp path] (2.640,7.554)--(2.730,7.554);
\draw[gp path] (20.895,7.554)--(20.805,7.554);
\draw[gp path] (2.640,7.773)--(2.730,7.773);
\draw[gp path] (20.895,7.773)--(20.805,7.773);
\draw[gp path] (2.640,7.953)--(2.730,7.953);
\draw[gp path] (20.895,7.953)--(20.805,7.953);
\draw[gp path] (2.640,8.104)--(2.730,8.104);
\draw[gp path] (20.895,8.104)--(20.805,8.104);
\draw[gp path] (2.640,8.236)--(2.730,8.236);
\draw[gp path] (20.895,8.236)--(20.805,8.236);
\draw[gp path] (2.640,8.351)--(2.730,8.351);
\draw[gp path] (20.895,8.351)--(20.805,8.351);
\draw[gp path] (2.640,8.455)--(2.820,8.455);
\draw[gp path] (20.895,8.455)--(20.715,8.455);
\node[gp node right] at (2.272,8.455) {$10^3$};
\draw[gp path] (2.640,9.137)--(2.730,9.137);
\draw[gp path] (20.895,9.137)--(20.805,9.137);
\draw[gp path] (2.640,9.535)--(2.730,9.535);
\draw[gp path] (20.895,9.535)--(20.805,9.535);
\draw[gp path] (2.640,9.818)--(2.730,9.818);
\draw[gp path] (20.895,9.818)--(20.805,9.818);
\draw[gp path] (2.640,10.037)--(2.730,10.037);
\draw[gp path] (20.895,10.037)--(20.805,10.037);
\draw[gp path] (2.640,10.217)--(2.730,10.217);
\draw[gp path] (20.895,10.217)--(20.805,10.217);
\draw[gp path] (2.640,10.368)--(2.730,10.368);
\draw[gp path] (20.895,10.368)--(20.805,10.368);
\draw[gp path] (2.640,10.500)--(2.730,10.500);
\draw[gp path] (20.895,10.500)--(20.805,10.500);
\draw[gp path] (2.640,10.615)--(2.730,10.615);
\draw[gp path] (20.895,10.615)--(20.805,10.615);
\draw[gp path] (2.640,10.719)--(2.820,10.719);
\draw[gp path] (20.895,10.719)--(20.715,10.719);
\node[gp node right] at (2.272,10.719) {$10^4$};
\draw[gp path] (2.640,11.401)--(2.730,11.401);
\draw[gp path] (20.895,11.401)--(20.805,11.401);
\draw[gp path] (2.640,11.799)--(2.730,11.799);
\draw[gp path] (20.895,11.799)--(20.805,11.799);
\draw[gp path] (2.640,12.082)--(2.730,12.082);
\draw[gp path] (20.895,12.082)--(20.805,12.082);
\draw[gp path] (2.640,12.301)--(2.730,12.301);
\draw[gp path] (20.895,12.301)--(20.805,12.301);
\draw[gp path] (2.640,12.481)--(2.730,12.481);
\draw[gp path] (20.895,12.481)--(20.805,12.481);
\draw[gp path] (2.640,12.632)--(2.730,12.632);
\draw[gp path] (20.895,12.632)--(20.805,12.632);
\draw[gp path] (2.640,12.764)--(2.730,12.764);
\draw[gp path] (20.895,12.764)--(20.805,12.764);
\draw[gp path] (2.640,12.879)--(2.730,12.879);
\draw[gp path] (20.895,12.879)--(20.805,12.879);
\draw[gp path] (2.640,12.983)--(2.820,12.983);
\draw[gp path] (20.895,12.983)--(20.715,12.983);
\node[gp node right] at (2.272,12.983) {$10^5$};
\draw[gp path] (2.640,1.663)--(2.640,1.843);
\draw[gp path] (2.640,12.983)--(2.640,12.803);
\node[gp node center] at (2.640,1.047) {$1$};
\draw[gp path] (4.310,1.663)--(4.310,1.753);
\draw[gp path] (4.310,12.983)--(4.310,12.893);
\draw[gp path] (5.286,1.663)--(5.286,1.753);
\draw[gp path] (5.286,12.983)--(5.286,12.893);
\draw[gp path] (5.979,1.663)--(5.979,1.753);
\draw[gp path] (5.979,12.983)--(5.979,12.893);
\draw[gp path] (6.517,1.663)--(6.517,1.753);
\draw[gp path] (6.517,12.983)--(6.517,12.893);
\draw[gp path] (6.956,1.663)--(6.956,1.753);
\draw[gp path] (6.956,12.983)--(6.956,12.893);
\draw[gp path] (7.327,1.663)--(7.327,1.753);
\draw[gp path] (7.327,12.983)--(7.327,12.893);
\draw[gp path] (7.649,1.663)--(7.649,1.753);
\draw[gp path] (7.649,12.983)--(7.649,12.893);
\draw[gp path] (7.933,1.663)--(7.933,1.753);
\draw[gp path] (7.933,12.983)--(7.933,12.893);
\draw[gp path] (8.186,1.663)--(8.186,1.843);
\draw[gp path] (8.186,12.983)--(8.186,12.803);
\node[gp node center] at (8.186,1.047) {$10$};
\draw[gp path] (9.856,1.663)--(9.856,1.753);
\draw[gp path] (9.856,12.983)--(9.856,12.893);
\draw[gp path] (10.833,1.663)--(10.833,1.753);
\draw[gp path] (10.833,12.983)--(10.833,12.893);
\draw[gp path] (11.526,1.663)--(11.526,1.753);
\draw[gp path] (11.526,12.983)--(11.526,12.893);
\draw[gp path] (12.063,1.663)--(12.063,1.753);
\draw[gp path] (12.063,12.983)--(12.063,12.893);
\draw[gp path] (12.502,1.663)--(12.502,1.753);
\draw[gp path] (12.502,12.983)--(12.502,12.893);
\draw[gp path] (12.874,1.663)--(12.874,1.753);
\draw[gp path] (12.874,12.983)--(12.874,12.893);
\draw[gp path] (13.195,1.663)--(13.195,1.753);
\draw[gp path] (13.195,12.983)--(13.195,12.893);
\draw[gp path] (13.479,1.663)--(13.479,1.753);
\draw[gp path] (13.479,12.983)--(13.479,12.893);
\draw[gp path] (13.733,1.663)--(13.733,1.843);
\draw[gp path] (13.733,12.983)--(13.733,12.803);
\node[gp node center] at (13.733,1.047) {$100$};
\draw[gp path] (15.402,1.663)--(15.402,1.753);
\draw[gp path] (15.402,12.983)--(15.402,12.893);
\draw[gp path] (16.379,1.663)--(16.379,1.753);
\draw[gp path] (16.379,12.983)--(16.379,12.893);
\draw[gp path] (17.072,1.663)--(17.072,1.753);
\draw[gp path] (17.072,12.983)--(17.072,12.893);
\draw[gp path] (17.610,1.663)--(17.610,1.753);
\draw[gp path] (17.610,12.983)--(17.610,12.893);
\draw[gp path] (18.049,1.663)--(18.049,1.753);
\draw[gp path] (18.049,12.983)--(18.049,12.893);
\draw[gp path] (18.420,1.663)--(18.420,1.753);
\draw[gp path] (18.420,12.983)--(18.420,12.893);
\draw[gp path] (18.742,1.663)--(18.742,1.753);
\draw[gp path] (18.742,12.983)--(18.742,12.893);
\draw[gp path] (19.025,1.663)--(19.025,1.753);
\draw[gp path] (19.025,12.983)--(19.025,12.893);
\draw[gp path] (19.279,1.663)--(19.279,1.843);
\draw[gp path] (19.279,12.983)--(19.279,12.803);
\node[gp node center] at (19.279,1.047) {$1000$};
\draw[gp path] (2.640,12.983)--(2.640,1.663)--(20.895,1.663)--(20.895,12.983)--cycle;
\node[gp node right] at (6.956,8.854) {$\beta=20\%$};
\node[gp node right] at (15.011,6.191) {$\beta=20\%$};
\node[gp node right] at (18.801,3.927) {$\beta=33\%$};
\node[gp node center,rotate=-270] at (0.492,7.323) {Throughput (tps)};
\node[gp node center] at (11.767,0.431) {Latency (s)};
\draw[gp path] (3.008,1.843)--(3.008,3.691)--(10.180,3.691)--(10.180,1.843)--cycle;
\node[gp node right] at (7.792,3.383) {Algorand};
\gpcolor{rgb color={0.580,0.000,0.827}}
\gpsetlinewidth{3.00}
\draw[gp path] (8.160,3.383)--(9.812,3.383);
\draw[gp path] (4.702,7.861)--(6.600,8.288)--(7.566,8.575)--(8.592,8.555)--(8.583,8.554)%
  --(9.146,8.602)--(9.602,8.630)--(11.338,8.587)--(12.111,8.688)--(12.695,8.725)--(12.882,8.652)%
  --(13.235,8.687)--(13.514,8.695)--(13.827,8.702);
\gpsetpointsize{14.00}
\gppoint{gp mark 1}{(4.702,7.861)}
\gppoint{gp mark 1}{(6.600,8.288)}
\gppoint{gp mark 1}{(7.566,8.575)}
\gppoint{gp mark 1}{(8.592,8.555)}
\gppoint{gp mark 1}{(8.583,8.554)}
\gppoint{gp mark 1}{(9.146,8.602)}
\gppoint{gp mark 1}{(9.602,8.630)}
\gppoint{gp mark 1}{(11.338,8.587)}
\gppoint{gp mark 1}{(12.111,8.688)}
\gppoint{gp mark 1}{(12.695,8.725)}
\gppoint{gp mark 1}{(12.882,8.652)}
\gppoint{gp mark 1}{(13.235,8.687)}
\gppoint{gp mark 1}{(13.514,8.695)}
\gppoint{gp mark 1}{(13.827,8.702)}
\gppoint{gp mark 1}{(8.986,3.383)}
\gpcolor{color=gp lt color border}
\node[gp node right] at (7.792,2.767) {Longest Chain};
\gpcolor{rgb color={0.000,0.620,0.451}}
\draw[gp path] (8.160,2.767)--(9.812,2.767);
\draw[gp path] (13.909,3.053)--(15.031,5.914)--(15.310,7.071)--(16.828,8.001)--(17.271,8.378)%
  --(17.792,9.074)--(18.960,9.779)--(20.649,10.328);
\gpsetpointsize{16.00}
\gppoint{gp mark 2}{(13.909,3.053)}
\gppoint{gp mark 2}{(15.031,5.914)}
\gppoint{gp mark 2}{(15.310,7.071)}
\gppoint{gp mark 2}{(16.828,8.001)}
\gppoint{gp mark 2}{(17.271,8.378)}
\gppoint{gp mark 2}{(17.792,9.074)}
\gppoint{gp mark 2}{(18.960,9.779)}
\gppoint{gp mark 2}{(20.649,10.328)}
\gppoint{gp mark 2}{(8.986,2.767)}
\draw[gp path] (18.201,2.172)--(19.276,5.074)--(19.437,6.359)--(20.895,7.211);
\gppoint{gp mark 2}{(18.201,2.172)}
\gppoint{gp mark 2}{(19.276,5.074)}
\gppoint{gp mark 2}{(19.437,6.359)}
\gppoint{gp mark 2}{(20.895,7.211)}
\gpcolor{color=gp lt color border}
\node[gp node right] at (7.792,2.151) {Prism};
\node[gp node right,rotate=45] at (8.662,12.352) {$\beta=20\%$};
\gpcolor{rgb color={0.898,0.118,0.063}}
\gpsetlinewidth{4.00}
\gppoint{gp mark 1}{(8.829,12.693)}
\gpcolor{color=gp lt color border}
\node[gp node right,rotate=45] at (10.055,12.372) {$\beta=33\%$};
\gpcolor{rgb color={0.898,0.118,0.063}}
\gppoint{gp mark 1}{(10.222,12.713)}
\gpcolor{color=gp lt color border}
\node[gp node right,rotate=45] at (11.276,12.392) {$\beta=40\%$};
\gpcolor{rgb color={0.898,0.118,0.063}}
\gppoint{gp mark 1}{(11.443,12.733)}
\gpcolor{color=gp lt color border}
\node[gp node right,rotate=45] at (12.144,12.450) {$\beta=42\%$};
\gpcolor{rgb color={0.898,0.118,0.063}}
\gppoint{gp mark 1}{(12.311,12.791)}
\gpcolor{color=gp lt color border}
\node[gp node right,rotate=45] at (13.177,12.369) {$\beta=43\%$};
\gpcolor{rgb color={0.898,0.118,0.063}}
\gppoint{gp mark 1}{(13.344,12.710)}
\gpcolor{color=gp lt color border}
\node[gp node right,rotate=45] at (16.180,12.346) {$\beta=44\%$};
\gpcolor{rgb color={0.898,0.118,0.063}}
\gppoint{gp mark 1}{(16.347,12.687)}
\gppoint{gp mark 1}{(8.986,2.151)}
\gpcolor{color=gp lt color border}
\gpsetlinewidth{1.00}
\draw[gp path] (2.640,12.983)--(2.640,1.663)--(20.895,1.663)--(20.895,12.983)--cycle;
%% coordinates of the plot area
\gpdefrectangularnode{gp plot 1}{\pgfpoint{2.640cm}{1.663cm}}{\pgfpoint{20.895cm}{12.983cm}}
\end{tikzpicture}
%% gnuplot variables

      \caption{Performance of Prism, Algorand, and the longest chain protocol. Curves represent tradeoff. For Prism, $\epsilon \approx 10^{-9}$. For the longest chain protocol, $\epsilon \approx 10^{-5}$. For Algorand, the attack probability $\approx 10^{-9}$.}
    \end{figure}
  
  \heading{Scalability}
\begin{table}[]
	\centering
	\begin{tabular}{ c | r | c c c }
	\toprule
    \textbf{Property} & \textbf{\#Nodes} & 100 & 300 & 1000 \\
	  \midrule
	 \multirow{4}{*}{Degree $=4$}   & Diameter & 5 & 7 & 9 \\
	                                & Throughput (tps) & $7.2\times 10^4$ & $7.4\times 10^4$ & $7.4\times 10^4$ \\
	                                & Latency (s) & 40 & 58 & 67 \\
	                                & Forking & 0.119 & 0.117 & 0.112 \\
	 \hline
	 \multirow{4}{*}{Diameter $=5$} & Degree & 4 & 6 & 8 \\
	                                & Throughput (tps) & $7.2\times 10^4$ & $7.9\times 10^4$ & $7.9\times 10^4$ \\ 
	                                & Latency (s) &40 & 44 & 37 \\
	                                & Forking & 0.119 & 0.119 & 0.127 \\
        \bottomrule
	\end{tabular}
	\caption{Performance of Prism with scaling to larger number of nodes.}
	\label{table:scale}
	\end{table}    
	
	\heading{Performance when under attack}
	\begin{figure}
      \centering
      \subcaptionbox{Censorship attack}{\begin{tikzpicture}[gnuplot]
%% generated with GNUPLOT 5.2p7 (Lua 5.3; terminal rev. Nov 2018, script rev. 107)
%% Mon Oct 14 19:52:12 2019
\tikzset{every node/.append style={font={\fontsize{24.0pt}{28.8pt}\selectfont}}}
\path (0.000,0.000) rectangle (17.000,10.500);
\gpcolor{color=gp lt color border}
\gpsetlinetype{gp lt border}
\gpsetdashtype{gp dt solid}
\gpsetlinewidth{2.50}
\draw[gp path] (2.286,2.364)--(2.466,2.364);
\node[gp node right] at (1.844,2.364) {$4$};
\draw[gp path] (2.286,4.213)--(2.466,4.213);
\node[gp node right] at (1.844,4.213) {$5$};
\draw[gp path] (2.286,6.062)--(2.466,6.062);
\node[gp node right] at (1.844,6.062) {$6$};
\draw[gp path] (2.286,7.911)--(2.466,7.911);
\node[gp node right] at (1.844,7.911) {$7$};
\draw[gp path] (2.286,9.760)--(2.466,9.760);
\node[gp node right] at (1.844,9.760) {$8$};
\draw[gp path] (2.286,2.364)--(2.286,2.544);
\draw[gp path] (2.286,9.760)--(2.286,9.580);
\node[gp node center] at (2.286,1.625) {$0\%$};
\draw[gp path] (4.521,2.364)--(4.521,2.544);
\draw[gp path] (4.521,9.760)--(4.521,9.580);
\node[gp node center] at (4.521,1.625) {$5\%$};
\draw[gp path] (6.757,2.364)--(6.757,2.544);
\draw[gp path] (6.757,9.760)--(6.757,9.580);
\node[gp node center] at (6.757,1.625) {$10\%$};
\draw[gp path] (8.992,2.364)--(8.992,2.544);
\draw[gp path] (8.992,9.760)--(8.992,9.580);
\node[gp node center] at (8.992,1.625) {$15\%$};
\draw[gp path] (11.228,2.364)--(11.228,2.544);
\draw[gp path] (11.228,9.760)--(11.228,9.580);
\node[gp node center] at (11.228,1.625) {$20\%$};
\draw[gp path] (13.463,2.364)--(13.463,2.544);
\draw[gp path] (13.463,9.760)--(13.463,9.580);
\node[gp node center] at (13.463,1.625) {$25\%$};
\draw[gp path] (13.463,4.213)--(13.283,4.213);
\node[gp node left] at (13.905,4.213) {$30$};
\draw[gp path] (13.463,6.062)--(13.283,6.062);
\node[gp node left] at (13.905,6.062) {$60$};
\draw[gp path] (13.463,7.911)--(13.283,7.911);
\node[gp node left] at (13.905,7.911) {$90$};
\draw[gp path] (13.463,9.760)--(13.283,9.760);
\node[gp node left] at (13.905,9.760) {$120$};
\draw[gp path] (2.286,9.760)--(2.286,2.364)--(13.463,2.364)--(13.463,9.760)--cycle;
\node[gp node center,rotate=-270] at (0.591,6.062) {Throughput ($\times 10^4$ tps)};
\node[gp node center] at (7.874,0.517) {Adv. Fraction};
\node[gp node right] at (7.148,3.652) {Throughput};
\gpcolor{rgb color={0.580,0.000,0.827}}
\gpsetlinewidth{3.50}
\draw[gp path] (7.590,3.652)--(9.538,3.652);
\draw[gp path] (2.286,8.602)--(4.521,8.041)--(6.757,7.388)--(8.992,6.989)--(11.228,5.783)%
  --(13.463,5.551);
\gpsetpointsize{14.00}
\gppoint{gp mark 1}{(2.286,8.602)}
\gppoint{gp mark 1}{(4.521,8.041)}
\gppoint{gp mark 1}{(6.757,7.388)}
\gppoint{gp mark 1}{(8.992,6.989)}
\gppoint{gp mark 1}{(11.228,5.783)}
\gppoint{gp mark 1}{(13.463,5.551)}
\gppoint{gp mark 1}{(8.564,3.652)}
\gpcolor{color=gp lt color border}
\node[gp node right] at (7.148,2.913) {Latency};
\gpcolor{rgb color={0.000,0.620,0.451}}
\draw[gp path] (7.590,2.913)--(9.538,2.913);
\draw[gp path] (2.286,4.748)--(4.521,4.689)--(6.757,4.675)--(8.992,4.592)--(11.228,4.465)%
  --(13.463,4.873);
\gppoint{gp mark 2}{(2.286,4.748)}
\gppoint{gp mark 2}{(4.521,4.689)}
\gppoint{gp mark 2}{(6.757,4.675)}
\gppoint{gp mark 2}{(8.992,4.592)}
\gppoint{gp mark 2}{(11.228,4.465)}
\gppoint{gp mark 2}{(13.463,4.873)}
\gppoint{gp mark 2}{(8.564,2.913)}
\gpcolor{color=gp lt color border}
\gpsetlinewidth{2.50}
\draw[gp path] (2.286,9.760)--(2.286,2.364)--(13.463,2.364)--(13.463,9.760)--cycle;
%% coordinates of the plot area
\gpdefrectangularnode{gp plot 1}{\pgfpoint{2.286cm}{2.364cm}}{\pgfpoint{13.463cm}{9.760cm}}
\end{tikzpicture}
%% gnuplot variables
}
      \subcaptionbox{Balancing attack}{\begin{tikzpicture}[gnuplot]
%% generated with GNUPLOT 5.2p7 (Lua 5.3; terminal rev. Nov 2018, script rev. 107)
%% Mon Oct 14 19:52:12 2019
\tikzset{every node/.append style={font={\fontsize{24.0pt}{28.8pt}\selectfont}}}
\path (0.000,0.000) rectangle (17.000,10.500);
\gpcolor{color=gp lt color border}
\gpsetlinetype{gp lt border}
\gpsetdashtype{gp dt solid}
\gpsetlinewidth{2.50}
\draw[gp path] (1.547,2.364)--(1.727,2.364);
\node[gp node right] at (1.105,2.364) {$4$};
\draw[gp path] (1.547,3.843)--(1.727,3.843);
\node[gp node right] at (1.105,3.843) {$5$};
\draw[gp path] (1.547,5.322)--(1.727,5.322);
\node[gp node right] at (1.105,5.322) {$6$};
\draw[gp path] (1.547,6.802)--(1.727,6.802);
\node[gp node right] at (1.105,6.802) {$7$};
\draw[gp path] (1.547,8.281)--(1.727,8.281);
\node[gp node right] at (1.105,8.281) {$8$};
\draw[gp path] (1.547,9.760)--(1.727,9.760);
\node[gp node right] at (1.105,9.760) {$9$};
\draw[gp path] (1.547,2.364)--(1.547,2.544);
\draw[gp path] (1.547,9.760)--(1.547,9.580);
\node[gp node center] at (1.547,1.625) {$0\%$};
\draw[gp path] (3.694,2.364)--(3.694,2.544);
\draw[gp path] (3.694,9.760)--(3.694,9.580);
\node[gp node center] at (3.694,1.625) {$5\%$};
\draw[gp path] (5.841,2.364)--(5.841,2.544);
\draw[gp path] (5.841,9.760)--(5.841,9.580);
\node[gp node center] at (5.841,1.625) {$10\%$};
\draw[gp path] (7.988,2.364)--(7.988,2.544);
\draw[gp path] (7.988,9.760)--(7.988,9.580);
\node[gp node center] at (7.988,1.625) {$15\%$};
\draw[gp path] (10.135,2.364)--(10.135,2.544);
\draw[gp path] (10.135,9.760)--(10.135,9.580);
\node[gp node center] at (10.135,1.625) {$20\%$};
\draw[gp path] (12.282,2.364)--(12.282,2.544);
\draw[gp path] (12.282,9.760)--(12.282,9.580);
\node[gp node center] at (12.282,1.625) {$25\%$};
\draw[gp path] (12.282,2.364)--(12.102,2.364);
\node[gp node left] at (12.724,2.364) {$0$};
\draw[gp path] (12.282,3.597)--(12.102,3.597);
\node[gp node left] at (12.724,3.597) {$500$};
\draw[gp path] (12.282,4.829)--(12.102,4.829);
\node[gp node left] at (12.724,4.829) {$1000$};
\draw[gp path] (12.282,6.062)--(12.102,6.062);
\node[gp node left] at (12.724,6.062) {$1500$};
\draw[gp path] (12.282,7.295)--(12.102,7.295);
\node[gp node left] at (12.724,7.295) {$2000$};
\draw[gp path] (12.282,8.527)--(12.102,8.527);
\node[gp node left] at (12.724,8.527) {$2500$};
\draw[gp path] (12.282,9.760)--(12.102,9.760);
\node[gp node left] at (12.724,9.760) {$3000$};
\draw[gp path] (1.547,9.760)--(1.547,2.364)--(12.282,2.364)--(12.282,9.760)--cycle;
\node[gp node right] at (12.282,9.218) {Longest Chain Latency};
\node[gp node center,rotate=-270] at (15.303,6.062) {Confirmation Latency (s)};
\node[gp node center] at (6.914,0.517) {Adv. Fraction};
\node[gp node right] at (6.396,4.360) {Throughput};
\gpcolor{rgb color={0.580,0.000,0.827}}
\gpsetlinewidth{3.50}
\draw[gp path] (6.838,4.360)--(8.786,4.360);
\draw[gp path] (1.547,7.178)--(3.694,7.112)--(5.841,6.981)--(7.988,7.491)--(10.135,6.930)%
  --(12.282,6.850);
\gpsetpointsize{14.00}
\gppoint{gp mark 1}{(1.547,7.178)}
\gppoint{gp mark 1}{(3.694,7.112)}
\gppoint{gp mark 1}{(5.841,6.981)}
\gppoint{gp mark 1}{(7.988,7.491)}
\gppoint{gp mark 1}{(10.135,6.930)}
\gppoint{gp mark 1}{(12.282,6.850)}
\gppoint{gp mark 1}{(7.812,4.360)}
\gpcolor{color=gp lt color border}
\node[gp node right] at (6.396,3.621) {Latency};
\gpcolor{rgb color={0.000,0.620,0.451}}
\draw[gp path] (6.838,3.621)--(8.786,3.621);
\draw[gp path] (1.547,2.442)--(3.694,2.440)--(5.841,2.457)--(7.988,2.469)--(10.135,2.584)%
  --(12.282,2.719);
\gppoint{gp mark 2}{(1.547,2.442)}
\gppoint{gp mark 2}{(3.694,2.440)}
\gppoint{gp mark 2}{(5.841,2.457)}
\gppoint{gp mark 2}{(7.988,2.469)}
\gppoint{gp mark 2}{(10.135,2.584)}
\gppoint{gp mark 2}{(12.282,2.719)}
\gppoint{gp mark 2}{(7.812,3.621)}
\gpcolor{color=gp lt color border}
\gpsetlinewidth{2.50}
\draw[gp path] (1.547,9.760)--(1.547,2.364)--(12.282,2.364)--(12.282,9.760)--cycle;
\gpcolor{rgb color={0.000,0.620,0.451}}
\gpsetdashtype{gp dt 3}
\gpsetlinewidth{4.50}
\draw[gp path](1.547,8.774)--(12.282,8.774);
%% coordinates of the plot area
\gpdefrectangularnode{gp plot 1}{\pgfpoint{1.547cm}{2.364cm}}{\pgfpoint{12.282cm}{9.760cm}}
\end{tikzpicture}
%% gnuplot variables
}
      \caption{Performance when under attacks. Censorship attack: attacker mines empty transaction and proposer blocks. Balancing attack: attacker tries to balance votes between multiple proposer blocks on the same level.}
  \end{figure}
    
\end{block}



  %\begin{block}{References}

   % \nocite{*}
   % \footnotesize{\bibliographystyle{plain}\bibliography{poster}}

  %\end{block}

\end{column}

\separatorcolumn
\end{columns}
\end{frame}

\end{document}
